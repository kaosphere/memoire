% Changer l'option these pour memoire ou thesepararticles selon le besoin
\documentclass[letterpaper, twoside, 12pt, memoire, creativecommons, hyperref]{thETS} 
%\documentclass[letterpaper, oneside, 12pt,memoire]{thETS} 
%\documentclass[letterpaper, oneside, 12pt,thesepararticles]{thETS}

\usepackage{times}
\usepackage[pdftex]{graphicx} %Pour insérer des figures
\usepackage[round]{natbibETS} %Pour faire les citations du genre Autheur (Année)
% Pour faire une citation du genre Auteur (Année), utilisez la commande \citet

%%%%%%%%%%%%%%%%%%%%%%%%%%%%%%%%%%%%%%%%%%%%%%%%%%%%%%%%%%%%%%%%
\usepackage{multibib} %Pour faire la liste des références
\newcites{refs}{LISTE DE RÉFÉRENCES}
%%%%%%%%%%%%%%%%%%%%%%%%%%%%%%%%%%%%%%%%%%%%%%%%%%%%%%%%%%%%%%%%

\usepackage{amsmath}  % Symboles et fonctions mathématiques supplémentaires
\usepackage{amssymb}  % Symboles math\'eatiques
\usepackage{caption}
\usepackage{url} % Prise en charge des url pour les référence
\urlstyle{rm} % Hyphenation des références

%%%%%%%%%%%%%%%%%%%%%%%%%%%%%%%%%%%%%%%%%%%%%%%%%%%%%%%%%%%%%%%%%%%%%%%%%%%%%%%%%%%%%%%%%%%%%%%
%%
%%   IMPORTANT
%%
%%  Si vous créez un fichier PDF directement avec PDFLatex, et vous utilisez Acrobat Reader
%% pour faire l'impression, n'oubliez-pas de changer l'option <<Mise à l'échelle>> pour la valeur
%% <<Aucune>> pour que les marges soient imprimés correctement.
%%
%%%%%%%%%%%%%%%%%%%%%%%%%%%%%%%%%%%%%%%%%%%%%%%%%%%%%%%%%%%%%%%%%%%%%%%%%%%%%%%%%%%%%%%%%%%%%%%%%%%%

\listfiles


%------------------------------------------------------------------------------------------------------------------------------------------

% A décommenter et modifier si nécessaire dans le cas d'une maitrise
\diplome{
  DE LA\\
  MAÎTRISE EN GÉNIE DE LA PRODUCTION AUTOMATISEE\\
  M.Ing.
}

\title{utilisation de signaux électromyographiques pour le contrôle d'un bras robotique pour usagers de fauteuils roulants motorisés.}

\author{Alexandre JUMELINE}
\authorcopyright{Alexandre Jumeline}

\datesoutenance{``Date de soutenance''}

\datedepot{``Date du dépôt au Bureau des cycles supérieurs''}

\directeur{M. }{Rachid Aissaoui}{Département de Génie de la Production Automatisée}

%\directeur{Mme.}{Prénom Nom}{Nom du département et institution}

%\codirecteur{Mme.}{Prénom Nom}{département et institution}

\president{M.}{Prénom Nom}{département et institution}

\examinexterne{M.}{Prénom Nom}{département et institution}{}

\jury{Mme.}{Prénom Nom}{département et institution}{}

%------------------------------------------------------------------------------------------------------------------------------------------

\begin{document}

\pagenumbering{Roman}

%%%%%%%%%%%%%%%%%%%%%%%%%%%%%%%%%%%%%%%%%%%%%%%%%%%
% PAGE TITRE
%%%%%%%%%%%%%%%%%%%%%%%%%%%%%%%%%%%%%%%%%%%%%%%%%%%
\maketitle

%%%%%%%%%%%%%%%%%%%%%%%%%%%%%%%%%%%%%%%%%%%%%%%%%%%
% PRÉSENTATION JURY
%%%%%%%%%%%%%%%%%%%%%%%%%%%%%%%%%%%%%%%%%%%%%%%%%%%
\presentjury

%%%%%%%%%%%%%%%%%%%%%%%%%%%%%%%%%%%%%%%%%%%%%%%%%%%
% AVANT PROPOS
%%%%%%%%%%%%%%%%%%%%%%%%%%%%%%%%%%%%%%%%%%%%%%%%%%%
\begin{avantpropos}
Ce projet a été réalisé en partenariat avec la société Canadienne Kinova dans le but d'ajouter une interface de contrôle à leur bras robotique Jaco contrôlable par le biais d'un Joystick. 

Le but principal était de prendre la suite du projet de maîtrise de Véronique Maheu qui consistait en l'étude de la performance de différents classificateurs dans le cadre de la classification de mouvements à partir de huit canaux électromyographiques. 

Le projet devait aussi se servir des paramètres cepstraux du signal électromyographiques. Pendant la réalisation, il s'est avéré que des caractéristiques du signal beaucoup plus simples permettent de discriminer suffisamment les mouvements effectués par l'utilisateur. De ce fait l'utilisation des paramètres cepstraux du signal a été abandonnée au profit d'une détection d'activité musculaire simple.
\end{avantpropos}

%%%%%%%%%%%%%%%%%%%%%%%%%%%%%%%%%%%%%%%%%%%%%%%%%%%
% REMERCIEMENTS
%%%%%%%%%%%%%%%%%%%%%%%%%%%%%%%%%%%%%%%%%%%%%%%%%%%
\begin{remerciements}
Lorem ipsum dolor sit amet, consectetur adipiscing elit. Pellentesque justo justo, porta sagittis feugiat eget, ornare rhoncus ligula. Nunc non odio sed lacus rutrum rhoncus. Mauris non congue arcu. Cras quis quam tortor. In ultrices tincidunt magna sed suscipit. Curabitur vel tellus sapien, ut tincidunt arcu. Maecenas dapibus ullamcorper urna, ut mollis mi tincidunt a. Nam eu orci nec lacus consectetur commodo. Donec purus tellus, consectetur at feugiat quis, scelerisque congue nibh. Aliquam urna dolor, congue nec euismod eget, convallis vitae libero. Sed vel magna suscipit leo suscipit porta quis et nunc. Nullam ante tellus, tincidunt a fringilla vel, rutrum non tellus. In volutpat consectetur purus, in euismod lorem feugiat vel. Aliquam sodales nisl eget sapien ullamcorper posuere consectetur orci bibendum. Vestibulum pulvinar viverra auctor. Vivamus ac sem et enim sodales dictum.
\end{remerciements}

%%%%%%%%%%%%%%%%%%%%%%%%%%%%%%%%%%%%%%%%%%%%%%%%%%%
% SOMMAIRE
%%%%%%%%%%%%%%%%%%%%%%%%%%%%%%%%%%%%%%%%%%%%%%%%%%%
\begin{sommaire}{Kinova, jaco, contrôleur EMG, temps-réel, blessé médullaire}

Les blessés médullaires de haut niveau lésionnel n’ont plus l’usage de leurs bras, mais conservent des capacités de mouvements résiduelles variant selon la hauteur de leur lésion. Ces capacités résiduelles concernent des mouvements restreints des doigts et des épaules principalement. Suivant le degré d’atteinte et la hauteur de la lésion de la moelle épinière, certaines personnes ne conservent pas suffisamment de liberté de mouvement dans leurs doigts pour utiliser ne serait-ce qu’un simple joystick. Pour ces personnes, les activités de la vie quotidienne (AVQ) deviennent des tâches très compliquées. 

La société canadienne Kinova a conçu un bras robotique, JACO, adaptable à un fauteuil roulant motorisé afin d’aider les personnes à mobilité réduite à effectuer leurs actions de la vie quotidienne plus facilement. En effet, celui-ci permet la saisie d’objets ou d’ouvrir des portes de manière intuitive en n’utilisant qu’un Joystick. Cependant les personnes n’ayant plus suffisamment d’activité musculaire dans les doigts ne peuvent donc pas profiter de ce progrès. Le but de ce projet est d’adapter l’utilisation du bras JACO aux personnes blessées médullaires en utilisant les signaux électromyographiques (EMG) de 4 muscles résiduels au niveau du cou et de l’épaule. 

L’acquisition des signaux EMG est faite avec des électrodes Delsys filtrées et pré-amplifiées en temps-réel à 2 kHz. Les électrodes sont placées sur quatre muscles présent chez les blessés médullaires de haut niveau : les deux trapèzes ainsi que les deux sternocléïdomastoïdiens. Ces muscles ont été choisis car ils permettent par leur seule activation de reconnaitre un mouvement donné : une élévation de l’épaule dans le cas des trapèzes et une rotation de la tête dans le cas des sternocléïdomastoïdiens.

L’énergie de Teager-Keiser est utilisée afin de déterminer si une activité musculaire est présente ou non, et chaque mouvement détecté peut être relié à une action du bras robot. Une calibration peut être effectuée pour chaque muscle afin de définir des seuils de détection personnalisés. Enfin un système de filtrage prédicitif  temps réel a été intégré afin d’augmenter la sensibilité du système a des mouvements plus lents ou pour les personnes ayant moins de tonus musculaire.

Ce projet constitue une avancée supplémentaire dans l’aide technique à la manipulation apportée aux blessés médullaire dans l’accomplissement de leurs tâches de la vie quotidienne. 



\end{sommaire}

%%%%%%%%%%%%%%%%%%%%%%%%%%%%%%%%%%%%%%%%%%%%%%%%%%%
% ABSTRACT
%%%%%%%%%%%%%%%%%%%%%%%%%%%%%%%%%%%%%%%%%%%%%%%%%%%
\begin{abstract}{titre original}{Kinova, jaco, EMG control, real time, spinal cord injury }

High level spinal cord injury victims don't have the capacity to use their arms, but still have control of several muscles depending on the level of injury. Those residual capacities are mostly finger and shoulder movements. Depending of the level of the injury, some people don't have enough residual muscular activity to move their fingers in order to control a simple joystick. For those people, daily life activities can be very hard to achieve. 

The Canadian corporation Kinova has developped a 7 degrees of freedom robotic arm, JACO, meant to be mounted on a motorised wheelchair in order to help spinal cord injury victims in their everyday's life. JACO can be used to grab objects or open doors only using a simple joystick. However, some poeople don't have enough musclar capacities to use this joystick, and therefor, can't use the arm.  The goal of this project is to adapt the use of JACO to spinal cord injury victims using electromyographic (EMG) signals sampled on 4 muscles of the neck and shoulders. 

EMG signal acquisition is performed using preamplified and filtered Delsys electrodes in real time at a sampling frequency of 2KHz. These electrodes are placed on 4 muscles still used by high level spinal cord injury victims : trapezius and sternocleidomastoid muscles. Those muscles were chosen beacause of their action in simple movements : shoulder elevation for trapeziuses and head rotation for sternocleidomastoids.

Teager-Keiser energy (TKE) is used in order to determine if a muscular onset is performed, and each move of the person can be linked to an action of the arm. A calibration can be done for each muscle in order to set a personalised detection threshold for onset detection. Finally, a real time predictive filter has been added in order to increase the sensitivity of the whole system and to be able to detect smaller and slower movements.

This project represent a step towards technical assistance brought to spinal cord injury victims and the improvement of their living conditions.


\end{abstract}

%%%%%%%%%%%%%%%%%%%%%%%%%%%%%%%%%%%%%%%%%%%%%%%%%%%
% TABLE DES MATIÈRES
%%%%%%%%%%%%%%%%%%%%%%%%%%%%%%%%%%%%%%%%%%%%%%%%%%%
\tableofcontents

%%%%%%%%%%%%%%%%%%%%%%%%%%%%%%%%%%%%%%%%%%%%%%%%%%%
% LISTE DES TABLEAUX
%%%%%%%%%%%%%%%%%%%%%%%%%%%%%%%%%%%%%%%%%%%%%%%%%%%
\listoftables

%%%%%%%%%%%%%%%%%%%%%%%%%%%%%%%%%%%%%%%%%%%%%%%%%%%
% LISTE DES FIGURES
%%%%%%%%%%%%%%%%%%%%%%%%%%%%%%%%%%%%%%%%%%%%%%%%%%%
\listoffigures

%%%%%%%%%%%%%%%%%%%%%%%%%%%%%%%%%%%%%%%%%%%%%%%%%%%
% LISTE DES ABBREVIATIONS
%%%%%%%%%%%%%%%%%%%%%%%%%%%%%%%%%%%%%%%%%%%%%%%%%%%
\begin{listofabbr}[3cm]
\item [EMG] Électromyographique
\item [BM] Blessé médullaire
\item [AVQ] Activité de la vie quotidienne
\item [TKE] Énergie de Teager-Keiser
\item [ETS] École de Technologie Supérieure
\item [CRC] Contrôle de redondance cyclique
\item [SOF] Start of frame
\item [EOF] End of frame

\end{listofabbr}


%%%%%%%%%%%%%%%%%%%%%%%%%%%%%%%%%%%%%%%%%%%%%%%%%%%
% LISTE DES SYMBOLES
%%%%%%%%%%%%%%%%%%%%%%%%%%%%%%%%%%%%%%%%%%%%%%%%%%%
\begin{listofsymbols}[3cm]
\item [$\textbf{EMG}$] Électromyographique
\item [$\alpha$] FAM choice hyperparameter
\item [$B_t$] New batch learning data block available that combines all available data up to a time $t$ (\emph{i.e.},$D_1 \cup ... \cup D_t$)
\item [$\beta$] FAM learning hyperparameter
\item [$\mathbf{c}$] Class input associated to $\textbf{a}$ for supervised learning
\item [$C_k$] Label of class $k$
\item [$|C_k|_\mathbf{LTM}$] Maximal number of samples per class in the long term memory
\item [$D_t$] New incremental learning data block available at a time $t$
\item [$D_t^\text{t}$] Training data set at a time $t$
\item [$D_t^\text{v}$] Validation data set at a time $t$
\item [$D_t^\text{f}$] Fitness estimation data set at a time $t$
\item [$\delta_{e_1e_2}$] Particle diversity between two ensemble members $e_1$ and $e_2$
\item [$\Delta$] Distance from a local best particle within which no personal best value can be memorized by other particles among the swarm
\item [$\Delta\theta_{e_1e_2}$] Diversity between two classifiers $e_1$ and $e_2$ determined with a FAM specific indicator
\item [$e_1,e_2$] Two ensemble members
\item [$\textit{EoFAM}$] An ensemble of fuzzy ARTMAP networks
\item [$\epsilon$] FAM match-tracking hyperparameter
\item [$f(\textbf{h})$] Objective function for a fuzzy ARTMAP hyperparameter vector $\textbf{h}$ in a static optimization environment
\item [$f(\textbf{h},t)$] Objective function for a fuzzy ARTMAP hyperparameter vector $\textbf{h}$ and at a time $t$ in a dynamic optimization environment
\item [$f_e(\mathbf{h},t)$] Objective function defined by the generalization error rate.
\item [$f_o(\textbf{h}_n,t)$] Objective function $o$ (during MOO)
\item [$f_s(\mathbf{h},t)$] Objective function defined by the size of the $F_2$ layer (\emph{i.e.}, number of $F_2$ layer nodes)
\item [$F^{ab}$] FAM map field
\item [$F_1$] FAM input layer
\item [$F_2$] FAM competitive hidden layer
\item [$\textit{FAM}_\text{estimation}$] FAM network used to estimate fitness with the data set $D_t^\text{f}$
\item [$\textit{FAM}_n$] FAM network associated to the best position of particle $n$
\item [$\textit{FAM}_{n,o}$] FAM network associated to the best position of particle $n$ for the objective $o$ (during MOO)
\item [$\textit{FAM}_\text{optimal}$] FAM network with the highest accuracy obtained after optimization on a learning block $D_t$
\item [$\textit{FAM}_n^\text{start}$] FAM network that defines the initial state of the particle $n$ prior learning data block $D_t$. During mono-objective optimization it corresponds to the best position of particle $n$, while it is associated with the current position of particle $n$ during MOO.
\item [$\textit{FAM}_\text{temp}$] Temporary fuzzy ARTMAP network used during fitness estimation

\end{listofsymbols}

\cleardoublepage

%%%%%%%%%%%%%%%%%%%%%%%%%%%%%%%%%%%%%%%%%%%%%%%%%%%
% CORPS DE LA THÈSE
%%%%%%%%%%%%%%%%%%%%%%%%%%%%%%%%%%%%%%%%%%%%%%%%%%%
\pagenumbering{arabic}

\reversemarginpar % pour que les marginpar s'amenent a GAUCHE du doc.

\begin{introduction}

Avec l’avancée des technologies de l’information, le domaine du génie des technologies de la santé est aujourd’hui en mesure de venir en aide aux personnes présentant un handicap naturel, ou résultant d’une pathologie ou d’un accident, en procurant à celles-ci des dispositifs permettant l’amélioration de leurs conditions et leur qualité de vie. Ces dispositifs sont appelés les aides techniques.

Ce projet s’intéresse principalement aux aides techniques de type robotique développées pour aider les personnes paralysées et en fauteuils roulants motorisés à effectuer leurs activités de la vie quotidienne (AVQ).L'une d'entre elles, le bras robotique Jaco a été développé par la société canadienne Kinova, et est contrôlé par un joystick de manière à ce que son utilisateur puisse facilement effectuer des actions simples, comme l’ouverture d’une porte ou la préhension d’un objet.

Les blessés médullaires (BM) de haut niveau lésionnel n’ont plus l’usage de leurs bras, mais conservent des capacités de mouvements résiduelles variant selon la hauteur de leur lésion. Ces capacités résiduelles concernent des mouvements restreints des doigts et des épaules principalement. Suivant le degré d’atteinte et la hauteur de la lésion de la moelle épinière, certaines personnes ne conservent pas suffisamment de liberté de mouvement dans leurs doigts pour utiliser ne serait-ce qu’un simple joystick. Pour ces personnes, les AVQ deviennent des tâches très compliquées.

Le but du projet est donc de faire en sorte de diversifier les interfaces de contrôle du bras Jaco, pour permettre à des personnes incapables de le contrôler exclusivement au joystick de pouvoir avoir accès à cette avancée technologique majeure. Pour ce faire, l'utilisation du signal électromyographique (EMG) a été choisie.

Ce projet de maîtrise étant en grande partie de la recherche appliquée, une grande partie présentera la conception du système. 

Une première partie de ce mémoire sera consacrée au fait de situer le projet dans l'univers scientifique actuel. Une mise en contexte sera premièrement effectuée dans le chapitre \ref{CHmiseencontexte}, suivie d'une revue de littérature et enfin de la problématique centrale du projet.

La partie principale du mémoire sera alors présentée dans le chapitre \ref{CHmethodo} : la méthodologie. Ce chapitre expliquera les choix matériels, logiciels et mathématiques effectués tout au long du projet. Toutes les étapes de conceptions seront détaillées ainsi que les intéractions entre les différentes parties du système. 

Enfin, la dernière partie comprendra l'étape de validation du système conçu afin de déterminer l'efficacité du système et de valider la faisabilité d'un tel projet. Cette validation sera effectuée pars l'application d'un protocole de recherche et à travers une étude effectuée sur 10 sujets sains afin de récolter des données qualitatives permettant une courte étude statistique des performences du système.

En résumé cette étude présente la conception d'un système de contrôle d'une aide technique robotisée grâce aux signaux EMG d'une personne BM et à la validation d'une preuve de concept dans le but d'une potentielle commercialisation future par la societé Kinova.

%
%\begin{figure}[ht]
%	\centering
%	\fbox{
%		\includegraphics{Figures/logoets.jpg}
%	}
%	\caption{Test de longue légende : Lorem ipsum dolor sit amet, consectetur adipiscing elit. Pellentesque justo justo, porta sagittis feugiat eget, ornare rhoncus ligula. Nunc non odio sed lacus rutrum rhoncus. Mauris non congue arcu. Cras quis quam tortor. In ultrices tincidunt magna sed suscipit}
%	\label{fig:logoets}
%\end{figure}



\end{introduction}

%\begin{revuedelitterature}
%
%Décommenter pour section revue de littérature, pour les thèses par article.
%
%\end{revuedelitterature}

\chapter{mise en contexte}\label{CHmiseencontexte}

Les personnes BM sont aujourd'hui en mesure de pouvoir continuer à garder une autonomie partielle dans leur vie quotidienne grâce à la combinaison de plusieurs progrès technologiques. Les fauteuils roulants motorisés sont, par exemple, une part de ce progrès qui contribue à permettre au BM de garder une certaine mobilité, et sont le point d'encrage d'autres aides techniques dont fait partie le bras robotique Jaco.

Ce chapitre a pour but de présenter les conditions de vie des personnes BM d'aujourd'hui ainsi que les aides techniques fournies par l'industrie qui permettent d'améliorer leurs conditions quotidiennes de vie. Cette mise en contexte est nécessaire afin de discerner dans quelle mesure ce projet de maîtrise participe au processus d'aide aux AVQ des personnes BM.

%\begin{align}
%   \beta = 8
%\end{align}
%\begin{align}
%   \gamma = \alpha \times 3
%\end{align}


\section{Les Blessés Médullaires}

\section{Les aides techniques aux actions de la vie quotidienne (AVQ)}

\section{Le Signal Électromyographique (EMG)}

%Test de décallage d'énumérations
%\begin{enumerate}
% \item test 1
% \item test 2
%\end{enumerate}
%
%Test de décallage de puces
%
%\begin{itemize}
% \item test 1
% \item test 2
%\end{itemize}
%
%Vérification des références: \cite{Arica2002}.
%\\
%Vérification de href: \href{http://www.etsmtl.ca/Etudiants-actuels/Cycles-sup/Realisation-etudes/Guides-gabarits}{Link to Gabarit ETS}, Lorem ipsum dolor sit amet, consectetur adipiscing elit. Pellentesque justo justo, porta sagittis feugiat eget, ornare rhoncus ligula.
%\\
%Vérification de url: \url{http://www.etsmtl.ca/Etudiants-actuels/Cycles-sup/Realisation-etudes/Guides-gabarits}, Lorem ipsum dolor sit amet, consectetur adipiscing elit. Pellentesque justo justo, porta sagittis feugiat eget, ornare rhoncus ligula.
%\\
%Fig.~\ref{fig:logoets}, Lorem ipsum dolor sit amet, consectetur adipiscing elit. Pellentesque justo justo, porta sagittis feugiat eget, ornare rhoncus ligula. Nunc non odio sed lacus rutrum rhoncus. Mauris non congue arcu. Cras quis quam tortor. In ultrices tincidunt magna sed suscipit. Curabitur vel tellus sapien, ut tincidunt arcu. Maecenas dapibus ullamcorper urna, ut mollis mi tincidunt a. Nam eu orci nec lacus consectetur commodo. Donec purus tellus, consectetur at feugiat quis, scelerisque congue nibh. Aliquam urna dolor, congue nec euismod eget, convallis vitae libero. Sed vel magna suscipit leo suscipit porta quis et nunc. Nullam ante tellus, tincidunt a fringilla vel, rutrum non tellus. In volutpat consectetur purus, in euismod lorem feugiat vel. Aliquam sodales nisl eget sapien ullamcorper posuere consectetur orci bibendum. Vestibulum pulvinar viverra auctor. Vivamus ac sem et enim sodales dictum, 
%%
%\begin{figure}[ht]
%	\centering
%	\fbox{
%		\includegraphics{Figures/logoets.jpg}
%	}
%	%\caption{Logo de l'École de Technologie Supérieure. Tiré de \cite{ETS2010}}
%	\caption{Distribution of horizontal velocity and streamlines in the vicinity of the nacelle for (a-c) non-operating turbine and (d-f) operating turbine ($U_{ref}$ = 20 m/s, the black circle indicates the position of the anemometer)}
%\end{figure}
%
%
%\begin{table}[ht]
%	\caption{Petit tableau: Encore une fois, test de legende de tableau plus longue pour vérifier le comportement de la liste}
%		\begin{tabular}{|c|c|c|c|}
%		\hline
%			blá & blá & blá & blá \\
%			\hline
%		\end{tabular}
%	\label{tab:UnTableau}
%\end{table}

\chapter{Revue de Littérature}

%\begin{figure}[ht]
%	\centering
%	\fbox{
%		\includegraphics{Figures/logoets.jpg}
%	}
%	\caption{Logo de l'École de Technologie Supérieure. \newline Tiré de \cite{ETS2010}}
%\end{figure}
%
%\begin{figure}[ht]
%	\centering
%	\fbox{
%		\includegraphics{Figures/logoets.jpg}
%	}
%	\caption{Logo de l'École de Technologie Supérieure. \newline Tiré de \cite{ETS2010}}
%\end{figure}


La classification de mouvements grâce au signal EMG a été énormément étudiée dans la littérature. Toute fois, une partie seulement de ces études l'a traitée de manière temps réelle. Enfin un type d'application vient encore se démarquer dans ce sous ensemble : les applications embarquées. 

Par application embarquée, nous entendons un système fonctionnant de manière autonome sur une carte électronique pouvant être déplacée facilement. Il faut donc enlever a ce sous ensemble toutes les applications réalisées sur ordinateurs. 

La première partie de cette revue de littérature s’intéressera à la classification EMG en temps réel, tandis que la deuxième partie sera centrée sur les applications embarquées dont fait état la littérature dans son état actuel.

\section{Classification EMG}

L’analyse des signaux EMG et la détermination du mouvement effectuée à partir de ceux-ci se divise en plusieurs grandes sous-parties incluant chacune un algorithme bien précis. Chacune de ces parties peut être réalisée avec plusieurs algorithmes différents impliquant chacun une précision, et une facilité d’implémentation différente. Des compromis doivent donc être faits entre la précision engendrée par un algorithme et sa complexité d’implémentation et d’exécution qui est souvent inversement proportionnelle. 

La première étape de traitement des signaux EMG consiste à déterminer quand il y a une activité musculaire et quand il n’y en a pas. Celle-ci est appelé détection d’activité, ou détection d'onset, et c’est la première partie qui sera analysée. L’étape suivante est l’extraction des caractéristiques du signal qui pourront permettre ensuite au classificateur de déterminer quel mouvement a été effectué. Enfin, une analyse de différents classificateurs sera effectuée, se basant sur les mêmes critères que les autres algorithmes, à savoir la précision, la facilité d’implémentation et le temps d’exécution. 

Un point important est à souligner, et ce point constituera un premier point de cette étude : tous ces algorithmes mis bouts à bouts doivent, une fois le choix d’architecture effectué permettre de réaliser le traitement de quatre canaux EMG en temps réel.

\subsection{Fenêtrage du signal}

Le signal électromyographique est découpé à des fins d'analyses et pour extraire les caractéristiques de celui-ci. La longueur de ces fenêtres est exprimée en temps ou en nombre d'échantillons. Un pas d’incrément est également défini, afin de choisir si les fenêtres se recouvrent ou non. 

La littérature étudiée fait état de beaucoup de tailles de fenêtres et de pas d’incréments différents. La plupart sont fixés de manière heuristique. Cependant, \cite{Englehart2003} évoquent le fait que cette fenêtre ne doit pas être trop grande de manière à ce que le temps de cette fenêtre et de son temps de traitement n’excède pas les 300ms de latence maximum au-delà desquels la latence est ressentie par l’utilisateur. Ainsi, leur fenêtre est fixée à 250ms avec un pas d’incrément à 125ms de manière à garder une marge suffisante pour le temps de calcul. 

Une fenêtre trop petite est sujette au biais de variance généré par les variation du signal et de ces caractéristiques qui peuvent jouer sur la précision du classificateur utilisé.

Un compromis doit donc être fait sur la longueur des fenêtres utilisées entre la  précision du classificateur et son temps de réponse.

\subsection{Algorithmes de détection d'activité}

Une fois une fenêtre du signal acquise, la première étape est la détection d'activité musculaire, ou détection d'"onset". 

La détection d’onset est le principe qui consiste à détecter le début d’une activité musculaire sur le signal EMG de manière automatique.  Pour ce faire, plusieurs algorithmes semblent bien fonctionner en combinant à la fois, une bonne précision et un nombre d’opérations flottantes à réaliser restreint. Cette détection d’onset permet d’économiser les ressources processeurs afin  de n’extraire les caractéristiques du signal que lorsque c’est utile et surtout de délimiter les zones d’activité musculaire pour la détection de mouvement. 
Il existe ainsi plusieurs caractéristiques du signal qui peuvent être extraites afin de détecter un onset. Une première largement utilisée \citep{Chang1996, Englehart2003, Tenore2007}est la méthode du Zero Crossing Rate (ZCR). Elle calcule le nombre de fois où la courbe du signal repasse par zero. 

Un élément important est que \cite{Chang1996} ont développé cette méthode sur un DSP et semble fournir des résultats satisfaisants. Cette étude présente également la mise en place d’une dead zone sur l’algorithme de zero crossing, qui permet de réduire, voir de supprimer l’influence du bruit sur le ZCR.
Le ZCR est une méthode simple à implémenter et très peu couteuse en ressources processeur pour la raison qu’elle ne nécéssite qu’un nombre d’addition égal N-1, N étant le nombre d’échantillons dans la fenêtre analysée. 

\cite{Peleg2002} optent pour une méthode consistant à appliquer un seuil directement sur l’enveloppe du signal. Le signal est d’abord passé dans un filtre passe-haut à 30Hz, puis la valeur absolue du signal est faite, enfin, le signal est passé dans un filtre passe-bas à 2.5Hz. Cette méthode est une analogie de la méthode Mean Absolute Value (MAV), le filtre passe bas ayant pour but d’effectuer une moyenne des valeurs absolues du signal. Suite à cela, le seuil appliqué sur l’enveloppe est fixé comme étant 10 pourcents plus haut que la valeur de l’enveloppe de l’EMG au repos. 

\cite{Tenore2007} utilisent également la méthode MAV sans toutefois préciser la manière dont ils ont choisi leur seuil de détection. 

Une autre méthode simple à implémenter et peu couteuse en temps de calcul est la méthode de la Teager-Kaiser Energy (TKE) qui correspond à un taux de variation du signal. Elle a été utilisée par \cite{Maheu2011} comme une caractéristique du signal utilisée dans la classification, mais peut aussi être utilisée dans la détection d’onset comme l'ont montré \cite{li2007}. Elle ne constitue qu’une soustraction et deux multiplications par échantillon dans la fenêtre d’analyse. 

\subsection{Caractéristiques du signal EMG utilisées}

Une fois l’ctivité musculaire du signal EMG détectée, on cherche à déterminer quel mouvement a été effectué. Ce travail est effectué par un classificateur. Ce classificateur doit donc prendre des éléments de comparaison en compte pour déterminer l’action faite. Ces éléments doivent être significatifs du signal capté, et sont appelés des features, ou des caractéristiques. 
Les caractéristiques extraites des signaux EMG sont nombreuses, aussi nous nous concentrerons sur celles ayant les meilleurs résultats dans la littérature. Ainsi la caractéristique ayant le plus de succès semble être l’auto-régression \citep{Chang1996, Peleg2002, Maheu2011}.

L’auto-régression consiste à trouver les n+1 coefficients d’un polynôme d’ordre n qui, appliqué sur un signal aléatoire, permettra de retrouver une approche du signal réel. Plusieurs algorithmes permettent de retrouver ces coefficients autorégressifs (AR).

\cite{Peleg2002} utilisent une auto-régression d’ordre 11. Cet ordre « a été déterminé de manière heuristique » \citep{Peleg2002}.

\cite{Chang1996} ont choisi l'utilisation des coefficients AR mais également les coefficients cepstraux obtenus à partir des AR. Ils préconisent également de fixer l’ordre de l’AR à 4 en se basant sur la littérature et en justifiant par un compromis entre une bonne représentation du signal  et un temps de calcul court. 
L’obtention des coefficients cepstraux à partir des coefficients AR se fait grâce à une relation mathématique récursive simple dépendant de l’ordre de l'auto-régression réalisée. 


\subsection{Classificateurs utilisés dans la détection de mouvements}

Une fois les caractéristiques du signal calculées,  celles-ci sont entrées dans un classificateur qui, grâce à un apprentissage, permet de déterminer quel mouvement a été effectué par le sujet. Encore une fois, beaucoup d’algorithmes existent pour réaliser cette opération. Toutefois, beaucoup de ceux-ci sont très consommateurs en temps de calculs, comme les réseaux de neurones (RN), où les analyses linéaires discriminantes (LDA). 

Toute fois \cite{englehart1999} ont montré que l'importance du choix du classificateur pour la reconnaissance de mouvements à partir de signaux EMG est limitée et que les gains en termes de performances de classificateurs complexes tels que les RN sont négligeables par rapport aux performances d'un classificateur plus simple comme le LDA.

Ceci étant dit nous nous intéresserons donc plus particulièrement aux deux études principales mettant en œuvre des dispositifs temps réels embarqués.

\cite{Chang1996} utilisent dans leur application embarquée un algorithme utilisant les distances des maximums de vraisemblances (MMLD), tandis que \cite{Tenore2007} utilisent quant à eux un classificateur de type LDA.


\section{Systèmes Embarqués temps réels et EMG}

La littérature comprend très peu d'applications embarquées de traitement temps réel de signaux EMG. Ainsi un premier article de référence a été publié par \cite{Chang1996} et rapporte la réalisation du traitement de 4 électrodes EMG afin de contrôler un pointeur informatique dans le but de remplacer une souris. Cette étude fait référence dans notre projet car elle a été développée dans le but de venir en aide aux blessés médullaires de haut niveau. 

Deux études de la même époque traitent ensuite également de traitement embarqué des EMG. \cite{martinez1999} utilisent un dsp, sans toutefois préciser lequel, pour traiter un canal EMG tandis \cite{Prasad1996} ont conçu une carte à base de DSP implantée dans un ordinateur pour acquérir et traiter cinq canaux EMG.

Enfin l'étude la plus récente est celle de \cite{Tenore2007} qui présentent dans leur article avoir réalisé l'une des premières applications embarquées classifiant le signal EMG.

\subsection{Le traitement temps réel}

Le temps réel au sens strict du terme en informatique, signifie que l’on doit savoir à tout instant combien de temps prendront les différentes actions réalisées par un programme. Ce temps peut s'exprimer de plusieurs manières, les deux principales étant le temps en secondes et le nombre d'instructions élémentaires d'une unité de calcul (microprocesseur, microcontrôleur).

Dans notre cas, et comme fixé dans la littérature du contrôle par EMG, la contrainte principale est le temps de réaction et de calcul maximal à laquelle le système doit satisfaire. La littérature traitant de l’implémentation temps réelle de tels systèmes parle ainsi d’un délai de calcul maximal de 300 millisecondes afin que l’utilisateur humain ne ressente pas la latence du système \citep{Chu2007, Englehart2003}.

\subsection{Méthodes d'acquisition du signal}

La plupart des études utilisées pour cette revue utilisent des électrodes de surface pour faire leur reconnaissance de mouvements. Cependant, les manières de filtrer et d'échantillonner sont différentes à travers les études. Cette partie a pour but de les regrouper et de les comparer. 

La plupart des études observent une fréquence d’échantillonnage de 1KHz permettant d’obtenir une bande utile théorique (respectant le théorême de Shannon) pour les signaux EMG allant jusqu’à 500Hz \citep{Chang1996, Chu2007, Englehart2003, Tenore2007}. 
La plupart des signaux EMG bruts sont amplifiés avec des gains de l’ordre de 2000, sont passé dans des filtres passe bande entre 10 et 450 ou 500 Hz.

\cite{Peleg2002} ont toutefois opté pour une fréquence d’échantillonnage à 500Hz, avec un filtre anti repliement à 250Hz, alors que \cite{Crawford2005} ont opté pour une fréquence d’échantillonnage à 2048 Hz.
Lorsqu’elles sont mentionnées, la précision des ADC utilisés varie suivant les études, allant de 16 bits \citep{Englehart2003} à 12 bits \citep{Tenore2007, Fukuda2003, Prasad1996}.


\subsection{Matériel informatique utilisé dans la littérature}

La plupart des application trouvées dans la littérature n'ont pas une définition toujours précise de la notion de temps réel. La plupart des articles présentent un système exécuté sur des machines puissantes, sur des systèmes d'exploitation souvent non temps réels. 

\cite{Englehart2003} ont réalisé une implémentation temps réelle d’algorithmes de reconnaissance de mouvement à partir de signaux EMG. Cette réalisation a été réalisée sous matlab et son extension de programmation temps réel sur un PC distant.

Si l'on s'interesse plus particulièrement aux applications embarquées, \cite{Chang1996} ont réalisé un montage électronique basé sur un DSP TMS320C31 de chez Texas Instruments. C’est un processeur 32bits comprenant un cœur à virgule flottante tournant à une fréquence de 20MHz, avec quatre niveaux de pipe-line permettant d’exécuter une instruction à chaque cycle d’horloge. Cette architecture leur a permis d’obtenir un temps total de traitement du signal entre 140 et 160 ms ce qui satisfait à la condition évoquée au début de l’étude concernant le respect d’un temps de réponse maximum de 300 ms. Toutefois, cette étude ayant été réalisée en 1996, il existe aujourd’hui quantité de DSP plus rapide et plus efficaces.

Toute fois cette étude, ainsi que celles de \cite{martinez1999} et de \cite{prasad1996} ont aujourd'hui autour d'une quinzaine d'années et le matériel informatique utilisé alors a aujourd'hui beaucoup évolué.

\cite{Tenore2007} ont utilisé également un DSP de chez Texas Instruments : un TMS320VC5509A couplé à un ADC dédié de chez National Instruments. Ce DSP est un processor comprenant un cœur à virgule fixe exécutant des actions relativement simples, comme le calcul du ZCR du signal, le calcul de la longueur d’onde du signal et le MAV. Ce DSP leur a permis d'implémenter un classificateur LDA.


\chapter{Problématique et Objectifs du Projet}

\chapter{Méthodologie}\label{CHmethodo}

\section{Choix des Muslces}

Le premier choix important est de choisir quels muscles seront utilisés pour réaliser la commande. Dans un premier temps, le critère principal est que les muscles choisis doivent toujours être fonctionnels chez les blessés médullaires. Ainsi, les muscles choisis par \cite{Chang1996}, les trapèzes et les sterno-cléïdo mastoïdiens sont considérer comme un choix judicieux dans un premier temps. En effet, ces deux muscles sont innervés en partie par des nerfs passant par la seconde vertèbre cervicale, ainsi, la plupart des blessés médullaires conserveront au moins une activité résiduelle dans ces muscles. 

\begin{figure}
	\centering
	\fbox{
		\includegraphics[width=0.6\linewidth]{Figures/muscles.png}
	}
	\caption{Muscles sterno-cléïdo mastoïdien et muscles du cou.
	par Berichard (travail personnel d'après Gray's Anatomy) [GFDL (http://www.gnu.org/copyleft/fdl.html) ou CC-BY-SA-3.0-2.5-2.0-1.0 (http://creativecommons.org/licenses/by-sa/3.0)], via Wikimedia Commons}
	\label{fig:muscles}
\end{figure}


D'autres muscles à considérer seraient également le muscle occipito-frontal qui opère l'élévation des sourcils. Toute fois, il est peu probable que les personnes blessées médullaires apprécie le fait d'avoir des électrodes collées sur le front en permanence. Ce muscles ne représente donc pas une alternative viable pour le projet, de même que pour tous les autres muscles faciaux.

Une autre solution seraient les muscles fléchisseurs et extenseurs du poignet. Toute fois, ces muscles, sont innervés respectivement par les nerfs médian et radial dont les faisceaux passent par les dernières vertèbres cervicales ainsi que la première vertèbre thoracique. Ainsi, les blessés médullaire de haut niveau n'auront plus suffisamment, voir plus du tout d'activité résiduelle dans ces muscles.

Un point vient également appuyer la sélection des muscles trapèzes et sterno-cléïdo mastoïdiens et qui va influencer la manière de développer le projet : l'acquisition de signaux EMG sur ces muscles n'implique pas de "cross-talk" c'est à dire qu'on ne voit pas l'activité d'un muscle sur l'enregistrement des autres. Cette observation a été réalisée au début de la réalisation du projet, après avoir effectué des tests préliminaires visuels. Ainsi, en faisant des mouvements simples de rotation de la tête et d'élévation des épaules, on ne voit qu'un seul muscle s'activer. Cette considération permet de faire un choix drastique sur la complexité du projet : on choisit de ne monitorer que des mouvements unitaires simples n'activant qu'un seul muscle. 

Ainsi, on est capable d'établir un lien direct entre l'activité musculaire d'un muscle particulier et le mouvement réalisé. Lorsqu'une activité est détectée sur un muscle sterno-cléïdo mastoïdien, c'est qu'une rotation de la tête a été effectuée, et si une activité est détectée sur un trapèze, c'est une élévation de l'épaule qui a été réalisée.

Ce postulat permet de réduire le temps de calcul nécessaire au traitement des signaux en enlevant complètement l'étape de classification. Ce choix comporte des points négatifs et des points positifs. Un atout majeur est de retirer toute la charge de calculs nécessaire pour le classificateur, et de concentrer ces calculs sur la détection d'activité musculaire ainsi que sur le filtrage numérique du signal. En contrepartie, le nombre de mouvements reconnus par le systèmes est réduit à quatre mouvements unitaires : un pour chaque muscle. 

Cependant cette perte en diversité de mouvements pourra être en partie comblée par des stratégies de communication différentes pour ces mouvements. En effet, on pourra effectuer des mouvements prolongés ou bien des mouvements composés de plusieurs clics, ou activations courtes répétées. 


\section{Architecture Matérielle}\label{CHarchimat}

Une fois le choix des muscles arrêté, les premiers choix a effectuer seront les choix architecturaux, et dans un premier temps, ceux concernant l'architecture matérielle. Ce chapitre présente les choix matériels effectué pour la conception du système tout en argumentant pourquoi chaque choix a été fait et dans quel cadre.

\subsection{Électrodes}

Le LIO travaille depuis maintenant plusieurs années avec des électrodes construites par la compagnie Delsys. Ainsi 4 électrodes de type DE-2.3 ont été achetée pour le projet et se présentent comme sur la figure \ref{fig:de2.3}. 

\begin{figure}
	\centering
	\fbox{
		\includegraphics[width=0.42\linewidth]{Figures/de23.png}
	}
	\caption{Électrode Delsys DE-2.3.}
	\label{fig:de2.3}
\end{figure}

Ces électrodes, contrairement au modèle moins cher DE-2.1 sont pré-amplifiées avec un gain de $1000 V/V \pm 1\%$ et pré-filtrées sur une bande passante de $20-450 Hz \pm 10\%$. Ceci nous permet de limiter les étapes de filtrage analogique avant de faire entrer le signal de l'électrode dans le microcontrôleur. 

Les électrodes sont câblées avec du fil blindé, protégeant des interférences électromagnétiques, le blindage étant relié à la masse du circuit. Pour que ce blindage soit efficace, le connecteur utilisé doit être également de bonne qualité. Pour ce faire, des connecteur de marque LEMO sont utilisés. Ces connecteurs sont utilisés dans beaucoup de domaines critiques demandant des connecteurs résistants au temps et aux interférences. 

Ainsi les connecteurs LEMO femelle correspondants à ceux présents sur les câbles des électrodes ont été acquis afin de relier les électrodes au système.

\begin{figure}
	\centering
	\fbox{
		\includegraphics[width=0.42\linewidth]{Figures/delsysneutre.png}
	}
	\caption{Électrode Delsys neutre.}
	\label{fig:delsysneutre}
\end{figure}

Plusieurs électrodes neutres, comme présentée par la figure \ref{fig:delsysneutre}, ont également été acquises. Ces électrodes sont nécessaire car c'est la référence utilisée pour mesurer la tension acquise sur les muscles. Elles seront positionnées sur une partie la moins charnue possible, comme par exemple la pointe du coude de l'utilisateur. Elles seront connectées directement à la référence du convertisseur analogique-numérique du microcontrôleur.

\subsection{Microcontrôleur}



\subsection{Adaptation des Tensions}

Les électrodes étant alimentées en $+5V/-5V$, et le PIC32 fonctionnant à $3.3V$, il est nécessaire d’adapter les signaux des électrodes de $+5V/-5V$ à $0V/3.3V$ afin de ne pas créer de sur-tension dans le microcontrôleur. Pour ce faire, un montage composé de deux étages d'amplificateurs opérationnels est nécessaire.  Les amplificateurs utilisés pour cette réalisation sont des composants intégrés de type LM741CN. Ces composants, bien que peu récents ont l'avantage d'avoir un comportement aujourd'hui bien maîtrisé et connu. 

Le premier étage est un montage inverseur ayant un gain de $1/3$. Ce montage est présenté par la figure \ref{fig:inverseur}.

\begin{figure}
	\centering
	\fbox{
		\includegraphics[width=0.6\linewidth]{Figures/inverseur.png}
	}
	\caption{Étage inverseur utilisé dans l'adaptation de tension.}
	\label{fig:inverseur}
\end{figure}

La fonction de transfert de cet étage d'amplification peut être calculée par l'équation \ref{eq:inverseur}. Ainsi, on fixe respectivement les valeurs des résistances $R1$ et $R2$ à $11Kohm$ et $3.3Kohm$.

\begin{align}\label{eq:inverseur}
   Vout = -Vin \times ( \frac{R2}{R1} )
\end{align}

Cet étage permet théoriquement d'amener la tension entre $-5V$ et $+5V$ à une tension comprise entre $-1.5V$ et $+1.5V$. En pratique, à cause des variation de valeurs des composants, et des caractéristiques de l'amplificateur opérationnel utilisé (imperfections statiques, tension d'offset), la tension est ramenée entre $-1.75V$ et $1.5V$.

Le deuxième étage est un offset réalisé par un montage sommateur inverseur permettant de repasser la plage de tension dans un intervalle positif. Le montage est décrit par la figure \ref{fig:sommateur1}. 

\begin{figure}
	\centering
	\fbox{
		\includegraphics[width=0.6\linewidth]{Figures/sommateur.png}
	}
	\caption{Étage sommateur utilisé dans l'adaptation de tension.}
	\label{fig:sommateur1}
\end{figure}

La fonction de transfert de cet étage d'amplification peut être calculée par l'équation \ref{eq:sommateur1}. 

\begin{align}\label{eq:sommateur1}
   Vout = -R3 \times ( \frac{V1}{R1} + \frac{V2}{R2})
\end{align}

Soit, si $R1 = R2 = R3$ : 

\begin{align}\label{eq:sommateur2}
   Vout = -(V1 + V2)
\end{align}

Ainsi, V1 représente la sortie du premier étage d'amplification et V2 la tension d'offset à ajouter à V1 pour faire passer toute la plage de tension dans les tension négatives (en théorie donc -1.5V). Enfin le facteur $-1$ généré par la partie inverseuse du montage permet de repasser l'intervalle de tension dans les tensions positives uniquement.

Le plus important pour le signal final est de ne pas dépasser $3.3V$ pour les crêtes hautes et de ne pas passer sous les $0V$ pour les crêtes basses. Le signal doit également être centré par rapport à cette plage de tension, et donc être centré aux alentours de $1.75V$.

La figure \ref{fig:adapttension} présente le schéma électronique utilisé sur chaque canal EMG avant de le connecter au PIC32.

\begin{figure}
	\centering
	\fbox{
		\includegraphics[width=0.6\linewidth]{Figures/circuitadapt.png}
	}
	\caption{Circuit électronique d'adaptation de la tension de sortie des électrodes.}
	\label{fig:adapttension}
\end{figure}

Cependant il est à souligner que la valeur exacte des composants électronique varie suivant leur tolérance (5\% pour les composants utilisés ici). Ainsi le voltage correspondant au 0 peut ainsi varier (1.72V, 1.68V, …) et un offset sera donc présent sur les valeurs enregistrées par le convertisseur du PIC. 

Le schéma \ref{fig:adapttension} présente les valeurs de composants pratiques pour la réalisation du circuit. Celles-ci sont définies en tenant compte des imperfections statiques des amplificateurs et des variations des composants.


\subsection{Liaison à l'Ordinateur}

L'architecture matérielle présentée au chapitre \ref{CHarchimat}%TODO : changer ca
 a présentée la manière dont sont reliées les différentes composantes physiques du système. Cette partie a pour but de présenter la liaison série réalisée entre la carte électronique du PIC32 et l'ordinateur utilisé pour visualiser les signaux temps réels.

Dans un premier temps, les niveaux de tensions du signal série délivré par le PIC32 ne sont pas compatibles avec ceux utilisés et transmis par l'ordinateur. En effet, le PIC fournit un signal entre 0V et +3.3V alors que les signaux du port série d'un ordinateur sont compris entre +12V et -12V. 

Afin de faire la transition entre ces deux niveaux de tension, un composants électronique intégré est utilisé : un MAX3232, présenté sur la figure \ref{fig:max3232}.

\begin{figure}
	\centering
	\fbox{
		\includegraphics[width=0.6\linewidth]{Figures/max3232.png}
	}
	\caption{Circuit de câblage type du composant max3232, extrait de la documentation officielle du composant produit par \cite{MAXIM3232}.}
	\label{fig:max3232}
\end{figure}

Une fois la liaison physique réalisée, un protocole de transmission de données a été mis en place. En effet, les données arrivant sur le port série en RS-232 étant traitées octet par octet, il a fallu mettre en place une trame de données afin que celles-ci soit réceptionnées correctement, et permettant de vérifier que toutes les données reçues sont bien intègres.
Le début de trame (SOF, Start of Frame) est constitué de 4 octets fixes, ne changeant jamais de valeur. Ceci constitue une protection contre une mauvaise détection de début de trame. En effet, la même séquence de quatre octets est très improbable de se présenter en tant que données, et de ce fait on aura peu de chance de détecter un mauvais début de trame. Viennent ensuite les données. La longueur en octets des données à transmettre sur la liaison série est facilement configurable et modifiable à la fois dans le programme du PIC32 et dans le programme de l'application de monitorage sur l'ordinateur.
Ensuite, les deux octets suivants concernent le contrôle de redondance cyclique (CRC) qui permet de vérifier que la trame reçue n'a pas perdu d'information, ou n'a pas vu un de ses octets corrompu lors du transfert. Le CRC est une valeur sur 2 octets obtenue grâce à un calcul récursif simple dont l'opération mathématique principale est une division modulo 2 dont le reste est le CRC. %TODO : detailler CRC

Le CRC est calculé en prenant pour entrée uniquement la partie données de la trame. A la réception d'une nouvelle trame, le CRC des données est calculé, puis comparé au CRC transmis dans la trame. Si les deux CRC sont identiques, c'est que la trame est intègre, sinon, la trame est considérée comme non valide, et les données ne sont pas enregistrées, car corrompues.

Enfin, la fin de trame (EOF, End of Frame), est composé de 2 octets fixes qui ne changent pas non plus de valeur afin de détecter quand une fin de trame est réceptionnée. La figure \ref{fig:serialframe} présente le schéma global de la trame utilisée dans le cadre de la liaison série.

\begin{figure}
	\centering
	\fbox{
		\includegraphics[width=\linewidth]{Figures/serialframe.png}
	}
	\caption{Format de la trame utilisée dans le cadre de la liaison série.}
	\label{fig:serialframe}
\end{figure}

Cette trame est utilisée autant dans les données allant du PIC32 vers l'ordinateur que pour les commandes de calibration envoyée au PIC32 par l'ordinateur.

\subsection{Liaison à Jaco}

\subsection{Schéma matériel Global}

\section{Choix des Caractéristiques du Signal à utiliser et des Outils algorithmiques et Mathématiques à Utiliser}

\subsection{Découpage des fenêtres d'échantillons}

\subsection{Détection d'activité musculaire}

\subsubsection{Énergie de Teager-Keiser}

\subsubsection{Vote à la majorité}

\section{Calibrage}

\section{Filtrage}

\subsection{Description du Filtre}

\subsection{Calibrage du Filtre}

\section{Architecture Logicielle}

\subsection{Firmware}

\subsubsection{Fonctionnement et Configuration Globale du PIC32}

\subsubsection{Alogorithme global}

\subsection{Application de monitorage des signaux}

Afin de pouvoir observer facilement les résultats de chaque portion de calculs effectués par le microcontrôleur, le PIC32 communique avec l’ordinateur par le biais d’une liaison série. 
Une application pour l’ordinateur a donc été réalisée en C++ afin de visualiser les signaux en provenance de la carte. Cette interface utilise le framework Qt ainsi que la librairie Qwt permettant de tracer des graphiques. 
Le firmware et l’application sont codés de manière à pouvoir changer facilement de mode d’affichage en fonction des courbes à tracer. Ainsi, on peut choisir de tracer les courbes des signaux EMG, les courbes des TKE ou bien des activités musculaires.      

\begin{figure}
	\centering
	\fbox{
		\includegraphics[width=0.6\linewidth]{Figures/desktop1.png}
	}
	\caption{Capture d'écran de l'application de monitorage des signaux.}
	\label{fig:desktop1}
\end{figure}

On voit sur la figure \ref{fig:desktop1} qu’une activité est détectée sur chaque canal étudié, et on peut voir les signaux EMG visualisable sur la figure de gauche. Cette interface permet de pouvoir visualiser n’importe quelle variable utilisée dans le firmware. Ainsi, les calculs du TKE et d'activité musculaire peuvent également être visualisés en temps réel.
Un ajout à l’application a également été fait de manière à pouvoir enregistrer dans un fichier un canal EMG échantillonné à 2KHz. Ceci permet de pouvoir faire des tests sur ce même signal en traitement différé dans Matlab par exemple.
Cette interface met également en exergue la discrimination des mouvements possible en se basant uniquement sur la détection d’activité. En effet, les signaux EMG présentés ci-dessus sont ceux respectivement d’un trapèze gauche, d’un sternocléidomastoïdien gauche, d’un sternocléidomastoïdien droit et d’un trapèze droit acquis lors d’une contraction isométrique. Ainsi si une activité est détectée sur un des deux trapèzes, on est déjà capables de dire qu’un mouvement du bras droit ou gauche a été effectué. Il en va de même pour les muscles sternocléidomastoïdiens et une rotation de la tête à droite ou à gauche. 


\subsubsection{Choix des Librairies et des langages}

\subsubsection{Fonctionnement Global et Schéma UML du Projet}

\subsubsection{Lien avec Jaco et Mono}

\section{Validation du Système}

\subsection{Protocole de validation}

\chapter{Résultats}

\chapter{Discussion}

\begin{conclusion}
Texte de conclusion

\end{conclusion}


%%%%%%%%%%%%%%%%%%%%%%%%%%%%%%%%%%%%%%%%%%%%%%%%%%%
%  ANNEXE:
%%%%%%%%%%%%%%%%%%%%%%%%%%%%%%%%%%%%%%%%%%%%%%%%%%%
\appendix


\multiannexe % si on a plus d'une annexe
%\include{annexe1}
\chapter{Titre de l'annexe} 

S'il y lieu

\section{Première Section de l'Annexe}
<Texte à inserer> 

Lorem ipsum dolor sit amet, consectetur adipiscing elit. Pellentesque justo justo, porta sagittis feugiat eget, ornare rhoncus ligula. Nunc non odio sed lacus rutrum rhoncus. Mauris non congue arcu. Cras quis quam tortor. In ultrices tincidunt magna sed suscipit. Curabitur vel tellus sapien, ut tincidunt arcu. Maecenas dapibus ullamcorper urna, ut mollis mi tincidunt a. Nam eu orci nec lacus consectetur commodo. Donec purus tellus, consectetur at feugiat quis, scelerisque congue nibh. Aliquam urna dolor, congue nec euismod eget, convallis vitae libero. Sed vel magna suscipit leo suscipit porta quis et nunc. Nullam ante tellus, tincidunt a fringilla vel, rutrum non tellus. In volutpat consectetur purus, in euismod lorem feugiat vel. Aliquam sodales nisl eget sapien ullamcorper posuere consectetur orci bibendum. Vestibulum pulvinar viverra auctor. Vivamus ac sem et enim sodales dictum. Test citation \cite{Bha10}

Tests de figure en annexe.

%%%%%%%%%%%%%%%%%%%%%%%%%%%%%%%%%%%%%%%%%%%%%%%%%%%%%%%%%%%%%%%%%%%%%
% On a crée les environiments figurep et tablep pour que les figures et
% les tableaux de l'annexe n'apparaissent pas dans les listes de figures et tableaux
%

\begin{figureap}[ht]
	\fbox{ % cette commande est nécéssaire pour encadrer les figures
	\centering
		\includegraphics{Figures/logoets.jpg}
	}
	\caption{Logo de l'ÉTS dans l'annexe. Ici on va mettre un peu plus de texte pour voir comment va être la présentation de
	la légende dans ce cas.}
	\label{fig:testap}
\end{figureap}
\begin{center}
\begin{equation} 
2*x=4 
\end{equation}
\end{center}

\begin{tableap}[*ht]
	\caption{Un autre tableau. Ici on va rédiger un peu plus de texte pour vérifier si la légende sera bien placé.}
		\begin{tabular}{|c|c|c|c|c|c|c|c|}
		\hline
			{\bf titre} & {\bf titre} & {\bf titre} & {\bf titre} & {\bf titre} & {\bf titre} & {\bf titre} & {\bf titre} \\
	  \hline
			blá & blá & blá & blá & blá & blá & blá & blá \\
	  \hline
			blá & blá & blá & blá & blá & blá & blá & blá \\
	  \hline
			blá & blá & blá & blá & blá & blá & blá & blá \\
	  \hline
			blá & blá & blá & blá & blá & blá & blá & blá \\
	  \hline
			blá & blá & blá & blá & blá & blá & blá & blá \\
	  \hline
			blá & blá & blá & blá & blá & blá & blá & blá \\
	  \hline
		\end{tabular}
	\label{tab:tableau_annexe}
\end{tableap}

Test citation \ref{fig:testap}.

Lorem ipsum dolor sit amet, consectetur adipiscing elit. Pellentesque justo justo, porta sagittis feugiat eget, ornare rhoncus ligula. Nunc non odio sed lacus rutrum rhoncus. Mauris non congue arcu. Cras quis quam tortor. In ultrices tincidunt magna sed suscipit. Curabitur vel tellus sapien, ut tincidunt arcu. Maecenas dapibus ullamcorper urna, ut mollis mi tincidunt a. Nam eu orci nec lacus consectetur commodo. Donec purus tellus, consectetur at feugiat quis, scelerisque congue nibh. Aliquam urna dolor, congue nec euismod eget, convallis vitae libero. Sed vel magna suscipit leo suscipit porta quis et nunc. Nullam ante tellus, tincidunt a fringilla vel, rutrum non tellus. In volutpat consectetur purus, in euismod lorem feugiat vel. Aliquam sodales nisl eget sapien ullamcorper posuere consectetur orci bibendum. Vestibulum pulvinar viverra auctor. Vivamus ac sem et enim sodales dictum.
\begin{equation}
x = 42
\end{equation} 
Lorem ipsum dolor sit amet, consectetur adipiscing elit. Pellentesque justo justo, porta sagittis feugiat eget, ornare rhoncus ligula. Nunc non odio sed lacus rutrum rhoncus. Mauris non congue arcu. Cras quis quam tortor. In ultrices tincidunt magna sed suscipit. Curabitur vel tellus sapien, ut tincidunt arcu. Maecenas dapibus ullamcorper urna, ut mollis mi tincidunt a. Nam eu orci nec lacus consectetur commodo. Donec purus tellus, consectetur at feugiat quis, scelerisque congue nibh. Aliquam urna dolor, congue nec euismod eget, convallis vitae libero. Sed vel magna suscipit leo suscipit porta quis et nunc.

\subsection{Test}


Lorem ipsum dolor sit amet, consectetur adipiscing elit. Pellentesque justo justo, porta sagittis feugiat eget, ornare rhoncus ligula. Nunc non odio sed lacus rutrum rhoncus. Mauris non congue arcu. Cras quis quam tortor. In ultrices tincidunt magna sed suscipit. Curabitur vel tellus sapien, ut tincidunt arcu. Maecenas dapibus ullamcorper urna, ut mollis mi tincidunt a. Nam eu orci nec lacus consectetur commodo. Donec purus tellus, consectetur at feugiat quis, scelerisque congue nibh. Aliquam urna dolor, congue nec euismod eget, convallis vitae libero. Sed vel magna suscipit leo suscipit porta quis et nunc. Nullam ante tellus, tincidunt a fringilla vel, rutrum non tellus. In volutpat consectetur purus, in euismod lorem feugiat vel. Aliquam sodales nisl eget sapien ullamcorper posuere consectetur orci bibendum. Vestibulum pulvinar viverra auctor. Vivamus ac sem et enim sodales dictum.

\citerefs{Test} \citerefs{Test2}


%%%%%%%%%%%%%%%%%%%%%%%%%%%%%%%%%%%%%%%%%%%%%%%%%%%
% LISTE DE RÉFERENCES
%
% IMPORTANT:
% Pour que ça marche:
%   1. Compiler le document une fois
%   2. Rouler la commande: << bibtex refs >>...cliquer sur le fichier update_refs.bat sur Windows
%   3. Recompiler le document
%%%%%%%%%%%%%%%%%%%%%%%%%%%%%%%%%%%%%%%%%%%%%%%%%%%

%Changement d'interligne pour passer en simple pour les références et la bibliographie.

%\newpage
\singlespacing
%\nociterefs{*} %Ici vous devez inclure les références qui ne sont pas cités, ou étoiles pour toutes les réfs
\bibliographystylerefs{bibETS}
\addcontentsline{toc}{chapter}{LISTE DE RÉFÉRENCES}
\bibliographyrefs{refs} % à décommanter et indiquer la liste des fichiers bib
\onehalfspacing

%------------------------------------------------------------------------------------------------------------------------------------------


%%%%%%%%%%%%%%%%%%%%%%%%%%%%%%%%%%%%%%%%%%%%%%%%%%%
% BIBLIOGRAPHIE
%%%%%%%%%%%%%%%%%%%%%%%%%%%%%%%%%%%%%%%%%%%%%%%%%%%
\newpage
\singlespacing
%\nocite{*}
\bibliographystyle{bibETS}
\addcontentsline{toc}{chapter}{BIBLIOGRAPHIE}
\bibliography{biblio} % à décommanter et indiquer la liste des fichiers bib
\onehalfspacing

\end{document}
